%%%%HEADER
\documentclass[twocolumn]{article}
\usepackage[a4paper, margin=1in, columnsep=20pt]{geometry}
\usepackage{amsmath, amssymb, graphicx, hyperref}
\usepackage[most,skins,breakable]{tcolorbox}
\usepackage[symbol]{footmisc}
\usetikzlibrary{calc}
\usepackage{xcolor}
\usepackage{caption}
\usepackage{blindtext}
\usepackage{bookmark}
\usepackage{breqn}
\usepackage[backend=biber,style=numeric]{biblatex} % 
\addbibresource{../references.bib}

\def\reals{\mathbb{R}}
% Define the custom definition box and command
\newtcolorbox{mydefinition}[2][]{%
    text width=0.95\columnwidth,
    before=\vspace{1mm}, 
    after=\vspace{1mm}, 
    colback=gray!10, % Background color (light gray)
    colframe=black!70,  % Border color
    coltitle=gray!10,  % Title color
    fonttitle=\bfseries, % Title font style
    sharp corners,   % Box style
    left=2pt,
    breakable,
    right=2pt,
    top=2pt,
    bottom=2pt,
    enhanced jigsaw,
    title=Definition: {#1},         % Title passed as the first argument
    colupper=black,  % Ensure proper content handling
    pad at break*=2pc,
    overlay first and middle={
        \coordinate (A1) at ($(interior.south east) + (-10pt,5pt)$);
        \coordinate (C1) at ($(interior.south east) + (-6pt,7.5pt)$);
        \draw[fill=black!50] (A1) -- +(0,5pt) -- (C1) -- cycle;
    }
    }
    
\newcommand{\definition}[2]{%
    \noindent%
    \begin{mydefinition}[#1]%
        .#2%
    \end{mydefinition}%
    \noindent
}

\newtcolorbox{myexample}[2][]{%
    text width=0.95\columnwidth,
    before=\vspace{1mm}, 
    after=\vspace{1mm}, 
    colback=orange!3, % Background color (light gray)
    colframe=black!70,  % Border color
    coltitle=gray!10,  % Title color
    fonttitle=\bfseries, % Title font style
    sharp corners,   % Box style
    left=2pt,
    right=2pt,
    top=2pt,
    bottom=2pt,
    breakable,
    title=Intuition: {#1},         % Title passed as the first argument
    pad at break*=1pc,
    overlay first and middle={
        \coordinate (A1) at ($(interior.south east) + (-10pt,5pt)$);
        \coordinate (C1) at ($(interior.south east) + (-6pt,7.5pt)$);
        \draw[fill=black!50] (A1) -- +(0,5pt) -- (C1) -- cycle;
    }
}

\newcommand{\example}[2]{%
    \noindent%
    \begin{myexample}[#1]%
    .#2%
    \end{myexample}%
    \noindent
}

\newtcolorbox{myalgorithm}[2][]{%
    text width=0.95\columnwidth,
    before=\vspace{1mm}, 
    after=\vspace{1mm}, 
    colback=blue!5, % Background color (light gray)
    colframe=black!70,  % Border color
    coltitle=gray!10,  % Title color
    fonttitle=\bfseries, % Title font style
    sharp corners,   % Box style
    left=2pt,
    right=2pt,
    top=2pt,
    bottom=2pt,
    breakable,
    title=Algorithm: {#1},         % Title passed as the first argument
    pad at break*=1pc,
    overlay first and middle={
        \coordinate (A1) at ($(interior.south east) + (-10pt,5pt)$);
        \coordinate (C1) at ($(interior.south east) + (-6pt,7.5pt)$);
        \draw[fill=black!50] (A1) -- +(0,5pt) -- (C1) -- cycle;
    }
}

\newcommand{\algorithm}[2]{%
    \noindent%
    \begin{myalgorithm}[#1]%
    .#2%
    \end{myalgorithm}%
    \noindent
}
%%%%HEADER
